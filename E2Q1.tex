              
\documentclass[a4paper, 11pt]{article}
\usepackage{nopageno}
\usepackage{fullpage}
\usepackage [english]{babel}
\usepackage [autostyle, english = american]{csquotes}
\MakeOuterQuote{"}
\begin{document}

\noindent
\large
\begin{flushright}
Tux Tucker \\
PHI 3670 \\ 
June 4, 2018 \\
\end{flushright}

%%INTRODUCTION%%
\begin{center}\section*{Could morality exist in the absence of God?}\end{center} 

This is a difficult question to answer without a clear definition of God or morality. For the purposes of this response I'll be using the definition of morality Shafer-Landau provides on page 66 of \textit{The Fundamentals of Ethics}: that morality is a set of norms, i.e. standards that we ought to live up to. For the definition of God, I will be assuming the generally Abrahamic view of an absolute, omnipotent, omniscient, omnipresent, omnibenevolent singular deity. I can't exactly present an objection to my view because I hesitate to claim one (especially as an ignostic/weak theological noncognitivist), but I will try to present an argument against the necessity of God for morality given these constraints. \\

According to certain theories, such as Divine Command theory, morality is impossible without God as its source. There are certainly appealing aspects to such a position: humans are fallible and our understanding of what is right is uncertain. By introducing God as the ultimate arbiter of morality, these questions are resolved on at least some level. In this way, moral realism grounded in theology can be a useful approach to many. But this position introduces its own problems, and I have doubts as to its truthfulness as well as its utility. \\

I will try to counter two of the more common versions of this claim: \\

%% CLAIM 1 %%
\noindent\textbf{Claim 1: Faith in God and adherence to his commands is the only source of morality} \\

This is a strong position and can be clarified further: \textit{active, intentional belief in one specific God and adherence to the his will as revealed through religious scripture is the only source of morality.} This view has gained popularity in certain contemporary Christian circles--particularly in American fundamentalism and evangelicalism. It can be traced back through the Third, Second, and First Great Awakenings which themselves had roots in the Protestant theology of Martin Luther in the 16th century and the Puritan ethos of the 16-17th century. The rise of the moral majority and the religious right of the 20th-21st century has revived it as a popular belief. \\

This view is problematic for multiple reasons. Firstly, it requires a kind of doxastic voluntarism--that is, it assumes that individuals have elective choice over their own beliefs. I have my doubts about this. Individuals may have some control over their beliefs, but certainly not the full control that makes \textit{Sola Fide} justifiable (which itself is misrepresented as an ethical claim when it's more appropriate to soteriology). I can offer my own experience here: when I left the faith, it was not because I chose to do so. Quite the opposite, as I had originally set out to grow closer to God by pursuing Truth! I had faith in the scriptures (Matthew 7:7-8, Luke 11:9) that my search for Truth could only lead to God, but the outcome of my search was entirely unexpected.  \\

My growing internal conflict was not merely a result of the "worldly knowledge" that I had sought despite being warned of (but how little faith must one have to hide from the world out of fear of being wrong!). The greatest conflicts were internal to both scripture and the religion as a whole and could not be reconciled. It's impossible to follow a literal reading of scripture when the scripture itself is inconsistent. One \textit{must} interpret the text, one \textit{must} choose an interpretation to follow. We are once again faced with problem of human error and inadequacy--something that the claim that "God is the perfect and objective source of morality" had sought to avoid. This is an irrecoverable blow to the fundamentalist claim that a literal interpretation of the Bible is the only way to gain true knowledge and the only true basis of morality. \\

Upon these revelations I was faced with a choice: the pursuit of Truth or the pursuit of God. But the pursuit of God had led me to truths that I didn't want to accept yet couldn't ignore. Any desperate faith remaining was disingenuous; I could no longer claim it while retaining my ethical and intellectual integrity. In this way I had little choice at all (again, the problem of doxastic voluntarism); I couldn't simply unlearn the things I had learned and return to a state of ignorance just as Adam and Eve couldn't un-eat from the Tree of Knowledge and return to the Garden. I had no choice but to seek truth. \\

In Martin Luther's own words: \\[-8pt]

\begin{centering}"I cannot and will not recant anything, for to go against conscience is neither right nor safe. Here I stand, I can do no other, so help me God. Amen."\end{centering} \\

If Claim 1 is true, then my actions were condemnable. But it doesn't seem right that an ethical theory would require one to abandon their conscience and integrity. And certainly an ethical theory that lays sole claim to the truth would be, at the very least, internally consistent. One might be tempted to make comparisons to the parable of the Wise and Foolish Builders found in Matthew 7:24-27--basing one's moral absolutes on unknowable and inconsistent rules is not unlike building a house on sand. For these reasons, I have confidence that Claim 1 cannot be correct. \\

%%CLAIM 2%%
\noindent\textbf{Claim 2: God is the only source of morality} \\

I have addressed the strong claim that faith in God and adherence to his commands is the only source of morality. However, there are still arguments to be made for God as the sole source of morality with a weaker version of this statement, which I will attempt to address here. \\

Many theologians claim that God is the necessary source of morality for all, regardless of one's particular faith or lack thereof. These arguments tend to claim that God has revealed himself to all regardless of whether he is recognized as God, and any goodness in the world has God as its ultimate source. There are certainly theologians and Christian philosophers who defend this position very well, such as St. Augustine and Thomas Aquinas. \\

Augustine's theology was deeply influenced by Platonism and Aquinas was just as deeply influenced by Augustine. This is clearly demonstrated in their similar metaphysics and epistemology; their view of God and nature borrows heavily from concepts such as Plato's Forms, which states that any physical object in our world is reflective of an absolute, abstract Ideal. For Aquinas and Augustine, God is this Ideal. Both of their philosophies equate goodness in some way with being: anything that has being has goodness, and this goodness has its source in God. Furthermore, a being acts ethically inasmuch as it fulfills its \textit{telos}--its purpose. In this context, this telos defines moral action as that which enables a human to use their uniquely human faculties towards the pursuit of the ultimate Good, which is God. Love of God and use of reason, therefore, are the highest ends a human can seek. \\

This seems like a much more tenable claim. Rather than insisting on any particular interpretation of doctrine, this view calls upon humans to invoke their senses and reason and by doing so, come to a greater love and understanding of God. In this way, it's much more compatible with both naturalist and analytic perspectives. Yet trouble still remains. Must it be the case that goodness, rightness, and moral behavior are entirely dependent on the existence of such a God? \\

The four cardinal virtues--prudence, justice, temperance, courage--are no more likely to be seen in a theist and no less likely to be found in an atheist. The problem of the virtuous pagan is one that both Augustine and Aquinas sought to resolve, although I believe they did so unsuccessfully. In order to maintain the claim that God is the source of all goodness and morality, it is necessary to qualify the atheist and their actions as somehow less good. One might claim that the atheist is not truly acting out of genuine charity because the source of their morality is flawed, being founded in something other than love for God. This seems unfair, but it's one of the few possibilities that allow for the ethical theory to hold. Another possibility would be that atheists can be good people who are capable of performing morally correct actions, but that while they may love their neighbor they are failing in their greater moral duty to love God. The atheist in this case may be virtuous in other ways, but they cannot be fully virtuous as they lack the virtue of faith. But if the atheist is in every other respect indistinguishable from the believer, what makes faith a virtue? How is it even necessary? \\

Perhaps it isn't necessary at all, and the claim that "God is the only source of morality" is flawed. If both Case 1 and Case 2 are flawed then we must look for moral guidance somewhere other than God. This brings me to my final claim. \\

%%CLAIM 3%%
\noindent\textbf{Claim 3: Morality can exist in the absence of God} \\

Can religion be a source of moral guidance? Possibly. But if this is the case then it doesn't necessitate God's existence. One can be a good person who commits to right action even if the reasons for such are not well grounded, and that applies regardless of one's faith status. Mirroring the problem of the "virtuous atheist" we are then confronted with the problem of the "virtuous believer". But is this really such a problem? It's perfectly plausible to construct a system of ethics that would consider the virtuous believer no different or lesser than the virtuous atheist. Utilitarianism and consequentialism could be considered such systems. Thus it can be seen that Claim 3 does not run into the same problems as Claim 2. \\

There is little reason to believe that morality cannot exist in the absence of God. Humans are, by nature, both pattern-seeking and social beings. We search for rule and structure and cannot disentangle ourselves from our natural and social environment. Empathy and a sense of right and wrong are, while not universal, generally inborn and innate aspects of the human experience. Nor are they limited to humans; altruistic behavior is found throughout the animal kingdom, as is "punishment" of an individual who fails to adhere to the social rules. If one is to abandon human anthropocentrism, we find that some primate societies have developed social structures that reward "good" behavior and punish "bad" behavior in ways that are not entirely unlike our own moral systems. Yet this view seems incomplete. If humans are innately kind, they are at least as innately cruel. This does not preclude a naturalist argument for morality, but it does complicate it. \\

The above marks the beginnings of a naturalist justification for morality in the absence of God, but naturalism is far from the only possibility. Shafer-Landau's defense of moral realism is decidedly non-naturalist (described, appropriately, in his book \textit{Moral Realism: A Defense}). Shafer-Landau's moral realism can be characterized by its endorsement of the stance-independence of moral reality, i.e. there exists moral truths that lie beyond human subjectivity. What is right and wrong, he claims, is both conceptually and existentially independent of human social agreements. He challenges non-cognitivism and constructivism as well as subjectivism and relativism--along with many other metaethical theories that allow for the absence of God--in defense of his own moral philosophy. Such a philosophy should be compelling to one who seeks external validation and objectivity in the realm of moral truth, such as those who might rely on God to fulfill such a role. He demonstrates that not only is God unnecessary for moral philosophy, but that it is entirely possible for something other than God to allow for a non-subjectivist, non-relativist ethics. 


\end{document}

              
